\documentclass[12pt]{beamer}
\mode<presentation>
\usetheme{Boadilla}
%\usepackage{beamerthemeGoettingen}
%\usepackage{beamerthemeDresden}

%%%%%%%%%%%%%%%%%%%%%%%%%%%%%%%%%%%%%%%%%%%%%%%%%%%%%%%%%%%%%%%%%%%%%%%%%%%%%%%%%%%%
% If true  then we have a longer  version for a 40 minutes talk
% If false then we have a shorter version for a 25 minutes talk
\newcommand{\longflag}{false}       
\newcommand{\includelongversion}[2]{\ifthenelse{\equal{\longflag}{true}}{#1}{#2}}
%%%%%%%%%%%%%%%%%%%%%%%%%%%%%%%%%%%%%%%%%%%%%%%%%%%%%%%%%%%%%%%%%%%%%%%%%%%%%%%%%%%%

\usepackage{graphicx}
\usepackage{latexsym}
\usepackage{amssymb, amsmath, stmaryrd}
\usepackage{hhline}
\usepackage{subfig}
\usepackage{multirow}
\usepackage{tikz}
\usepackage{minibox}
\usepackage{ifthen}
\usepackage{xspace}
\usepackage{mathabx}
\usepackage{url}
\usepackage{framed,color}

\usetikzlibrary{arrows,shapes,calc,decorations.pathreplacing}
\definecolor{pinegreen}{rgb}{0,0.55,0.45}

% don't bother me with fulsome warning messages
\tolerance=100000

\renewcommand{\topfraction}{.9}
\renewcommand{\bottomfraction}{.9}
\renewcommand{\textfraction}{0.1}

\definecolor{shadecolor}{rgb}{0.8,0.9,1.0}

\newcommand{\sys}[1]{\textsc{#1}}
\newcommand{\re}[1]{\ensuremath{\mathtt{#1}}}
\newcommand{\svar}[1]{\ensuremath{\left \llbracket #1 \right \rrbracket}}

\DeclareMathOperator{\pl}{\!+}

\newcommand{\ceil}[1]{\left\lceil#1\right\rceil}
\newcommand{\bexists}{\overline{\exists}}
\newcommand{\entails}{\vDash}
\newcommand{\essential}[1]{\left \llbracket #1 \right \rrbracket}
\newcommand{\fail}{\bf fail}
\newcommand{\lra}{\leftrightarrow}
\newcommand{\lu}{{L\overline{U}}}
\newcommand{\ra}{\rightarrow}
\newcommand{\fff}{{\cal F}}
\newcommand{\ttt}{{\cal T}}
\providecommand{\abs}[1]{\lvert#1\rvert}

\newcommand{\pset}[1]{{\cal P}(#1)}

%% Wrapping a tabular environment in a macro,
%% so it doesn't conflict with Tikz stuff.
\newcommand{\wtab}[1]{
  \begin{tabular}{c}
    #1
  \end{tabular}
}



%\xyoption{ps}
%\xyoption{color}
%\UseCrayolaColors
%\newcommand{\xyo}[1]{*+[o][F-]{#1}}
%\newcommand{\xyco}[2]{*+[o][F-:#1]{#2}}


%-----------------------------------------------------------------------%
\title[Introducing Wybe]{The Design and Implementation of Wybe}

%%%%%%%%%%%%%%%%%%%%%%%%%%%%%%%%%%%%%%%%%%%%%%%%%%%%%%%%%%%%%%%%
% Abstract:
%
% We present Wybe, a new language in the early stages of development.
% Wybe combines the best of declarative and imperative programming in
% a principled way.  It is intended to be easy to learn for beginner
% programmers, and also to scale up to large projects through good
% support for software engineering principles.
%%%%%%%%%%%%%%%%%%%%%%%%%%%%%%%%%%%%%%%%%%%%%%%%%%%%%%%%%%%%%%%%

\author[Peter Schachte]
{\textbf{Peter Schachte} }

\institute[Melbourne Uni]{The University of Melbourne \\ School of
  Computing and Information Systems}

\date{4 September 2019}

%% \ignore{
%%   \AtBeginSection[]
%%   {
%%      \begin{frame}<beamer>
%%      \frametitle{Outline}
%%      \tableofcontents[currentsection]
%%      \end{frame}
%%   }
%% }

\begin{document}

\frame{\titlepage}

%=======================================================================%
\section{Introduction}

%-----------------------------------------------------------------------%
\begin{frame}[fragile]
\frametitle{Motivation --- Wybe goals:}
\begin{itemize}
\item Be simple and easy to learn (but not necessarily orthodox)
\item Support declarative programming
\item Support imperative programming
\item Support best software engineering practice
\item Be efficient
\end{itemize}
\vspace{2ex}
\begin{quote}
  The lurking suspicion that something could be simplified is the
world's richest source of rewarding challenges. \\
\hspace*{1cm} --- Edsger Wybe Dijkstra
\end{quote}
\end{frame}

%=======================================================================%
\section{Interface Integrity}

%-----------------------------------------------------------------------%
\begin{frame}[fragile]
\frametitle{Interface Integrity}
\begin{itemize}
\item Major advantage of declarative languages:  
  no \emph{action at a distance} --- where change happens for no apparent reason
\item Caused by destructive update of aliased structures
\item Requires the programmer to have (and maintain!) a \emph{global}
  understanding of dynamic aliasing in program
\item Every function has two interfaces:
  \begin{itemize}
  \item \textbf{Apparent interface} appears in function declaration
    or call
  \item \textbf{Effective interface} is the information that passes into and out of function
  \end{itemize}
\item A function exhibits \textbf{interface integrity} if its apparent and
effective interfaces are identical --- no action at a distance

\end{itemize}

\end{frame}


%=======================================================================%
\section{Wybe Basics}

%-----------------------------------------------------------------------%
\begin{frame}[fragile]
\frametitle{Hello World}
\vspace*{-2ex}
\begin{itemize}
\item Like a scripting language, top-level statements in a module are executed
\item Top-level statements in a module are executed before top-level statements
  in any module that imports it
\item Hello World in Wybe:\vspace*{-1ex}
  \begin{minipage}{0.95\linewidth}
    \begin{block}{}
\begin{semiverbatim}
!println("Hello, World!")
\end{semiverbatim}
    \end{block}
  \end{minipage}
\item I'll explain the leading \texttt{!} a little later
\item Every Wybe source file is a module whose name is the file name with the
  \texttt{.wybe} extension removed.
\item Compiler is like make:  takes name of file to \emph{produce}\vspace*{-1ex}
  \begin{minipage}{0.95\linewidth}
    \begin{block}{}
\begin{semiverbatim}
\textbf{% }wybemk hello
\textbf{% }./hello
Hello, World!
\end{semiverbatim}
    \end{block}
  \end{minipage}
\end{itemize}
\end{frame}


%-----------------------------------------------------------------------%
\begin{frame}[fragile]
\frametitle{Information flow}
\begin{itemize}
\item Direction of information flow (\emph{mode}) is explicit
\item A bit like Ada's \texttt{in}, \texttt{out}, and \texttt{in out}
\item Unadorned variable name denotes variable value (call by value)
\item Question mark (\texttt{?}) in front of variable name indicates
  variable (re-)assignment (call by result)
\item Exclamation point (\texttt{!}) indicates both (call by value-result)
\item Which side of \texttt{=} code is on is irrelevant
\item \texttt{?x = x + 1} or \texttt{x + 1 = ?x} increments x
\item so does \texttt{incr(!x)}
\end{itemize}
\end{frame}


%-----------------------------------------------------------------------%
\begin{frame}[fragile]
\frametitle{Value semantics}
\begin{itemize}
\item Wybe has value semantics:
  \begin{itemize}
  \item Aliasing is not semantically significant
  \item Data structures cannot be mutated,
  \item But variables can be re-assigned
  \end{itemize}
\item Eg, \texttt{!pos.x = new\_x} assigns \texttt{pos}
  a \emph{new} position structure whose \texttt{x} member is
  \texttt{new\_x} and
  whose other members are the same as the original \texttt{pos}
\item Aliases of \texttt{pos} are not affected
\item Compile-time garbage collection:  when structure is unaliased,
  compiler implements this with mutation
\end{itemize}
\end{frame}


%-----------------------------------------------------------------------%
\begin{frame}[fragile]
\frametitle{Procedures}
\begin{itemize}
\item Same adornments are used in formal parameters
\item \texttt{def foo(w, x, ?y, !z) \{ \ldots \}} defines procedure with two
  inputs, one output, and one in-out parameter
\item Adornments in call must match definition (but see below\ldots)
\item Body of a procedure definition is a sequence of
  statements
\item There are a few built-in statement types, discussed below
\item Procedure calls are statements
\end{itemize}
\end{frame}


%-----------------------------------------------------------------------%
\begin{frame}[fragile]
\frametitle{If statement}
\begin{itemize}
\item Tests can be used in \texttt{if} statements
\item \texttt{if \{} \emph{test1}\texttt{::} \emph{statements} \ldots \\
\hspace*{1.5em} \texttt{$\mid$} \emph{test2}\texttt{::} \emph{statements} \ldots \\
\hspace*{1.5em} \ldots \\
\texttt{\}}
\item Tests are tried in order; body of first to succeed is executed
\item If none succeeds, none is executed
% \item Boolean expression \emph{e} is de-reified into the test \emph{e}
%   \texttt{= true}
\item \texttt{else} is a test that always succeeds
% \item Also an expression version of \texttt{if}, where
%   \emph{statements} are replaced with \emph{expressions}
% \item Tests of an \texttt{if} expression must be exhaustive
% \item Variables bound in tests are scoped to the corresponding statements
\item There will be a \texttt{case} statement with similar syntax
\end{itemize}
\end{frame}


%-----------------------------------------------------------------------%
\begin{frame}[fragile]
\frametitle{Loops}
\begin{itemize}
\item One modular looping construct:
  \texttt{do \{} \emph{loop-statements} \ldots \texttt{\}}
\item \emph{loop-statements} are any normal statements plus any
  special looping statements, including:
  \begin{itemize}
  \item \texttt{while} \emph{test} and \texttt{until} \emph{test}
    \begin{itemize}
    \item like conditional \texttt{break}
    \end{itemize}
  \item \texttt{when} \emph{test} and \texttt{unless} \emph{test}
    \begin{itemize}
    \item like conditional \texttt{continue}
    \end{itemize}
  % \item \texttt{for} \emph{generator}
  \end{itemize}
\item Include as many of these constructs as you like in the loop,
  wherever you like, \emph{e.g.:} \\
\begin{semiverbatim}
    do \{!print(prompt)
        !readln(?answer)
        until answer in ["y","n"]
        !println("Please answer 'y' or 'n'.")
    \}
\end{semiverbatim}
\end{itemize}
\end{frame}


%-----------------------------------------------------------------------%
\begin{frame}[fragile]
\frametitle{Expressions and Functions}
\begin{itemize}
\item Procedure call arguments can be expressions
\item Expression is just a procedure call with last argument omitted
\item The value of such an expression is the value that would be assigned
  to its omitted argument
\item \emph{E.g.,} \texttt{bar(x,y)} as an expression means call
  \texttt{bar(x,y,?temp)} and use \texttt{temp} as the value of
  \texttt{bar(x,y)}
\item \texttt{foo(bar(x,y),?z)} \quad means \quad
  \begin{minipage}[c]{0.4\linewidth}
  \texttt{bar(x,y,?temp)} \\
  \texttt{foo(temp,?z)}
  \end{minipage}
\item \texttt{def foo(x) = bar(x,x)} is syntactic sugar for \\
\texttt{def foo(x,?result) \{ bar(x,x,?result) \}}
\item Can use \texttt{foo} with either syntax regardless of which
  definition was used
\item A few built-in expressions like \texttt{let \{} \emph{stmts}
  \texttt{\} in} \emph{expr} and \emph{expr} \texttt{where \{} \emph{stmts}
  \texttt{\}}
% \item \texttt{a.b(c,\ldots)} is syntactic sugar for
%   \texttt{b(a,c,\ldots)}, to allow dot notation for member access and update
\end{itemize}
\end{frame}


%-----------------------------------------------------------------------%
\begin{frame}[fragile]
\frametitle{Reversibility}
\begin{itemize}
\item Procedures can be overloaded based on mode:
  \begin{itemize}
    \item \texttt{position(x, y, ?pos)}  constructs a position
    \item \texttt{position(?x, ?y, pos)}  deconstructs
  \end{itemize}
\item This is nicer as an expression (pattern matching)
  \begin{itemize}
  \item  \texttt{?pos = position(x,t)} constructs
  \item  \texttt{pos = position(?h,?t)} or deconstructs
  \end{itemize}
% \item \texttt{!pos.x = new\_x} replaces \texttt{x} of \texttt{pos} with
%   \texttt{new\_x}
% \item This is transformed to \quad
%   \begin{minipage}[c]{0.4\linewidth}
%   \texttt{x(pos,?temp)} \\
%   \texttt{x(!temp,y)} \\
%   \texttt{y(!x, temp)}
%   \end{minipage}
% \item Modes of all parameters but the last must uniquely determine the
%   mode of the last
% \item \texttt{def \_(x):} defines \texttt{\_} as a ``don't
%   care'' output
\end{itemize}
\end{frame}


%-----------------------------------------------------------------------%
\begin{frame}[fragile]
\frametitle{Modules}
\begin{itemize}
\item Each source file is a module
\item A directory of source files is also a module
\item A (sub)-module can be declared within a module
  \begin{itemize}
  \item Aliasing is not semantically significant
  \item Data structures cannot be mutated
  \item Variables can be re-assigned
  \end{itemize}
\item Eg, \texttt{x(!pos,new\_x)} assigns \texttt{pos}
  a \emph{new} position structure whose \texttt{x} member is \texttt{new\_x} and
  whose other members are the same as the original \texttt{pos}
\item Gives the feeling of changing values without action at a distance
\item So after \texttt{?old\_pos = pos ; x(!pos,new\_x)}, \texttt{old\_pos}
  still holds the old position, and \texttt{pos} holds the new one
\item Compile-time garbage collection:  when unique (unaliased),
  compiler transforms this (back) into destructive modification
\end{itemize}
\end{frame}


%-----------------------------------------------------------------------%
\begin{frame}[fragile]
\frametitle{Types}
\begin{itemize}
\item A Wybe type is just a (sub)module that's 
  \begin{itemize}
  \item Aliasing is not semantically significant
  \item Data structures cannot be mutated
  \item Variables can be re-assigned
  \end{itemize}
\item Eg, \texttt{x(!pos,new\_x)} assigns \texttt{pos}
  a \emph{new} position structure whose \texttt{x} member is \texttt{new\_x} and
  whose other members are the same as the original \texttt{pos}
\item Gives the feeling of changing values without action at a distance
\item So after \texttt{?old\_pos = pos ; x(!pos,new\_x)}, \texttt{old\_pos}
  still holds the old position, and \texttt{pos} holds the new one
\item Compile-time garbage collection:  when unique (unaliased),
  compiler transforms this (back) into destructive modification
\end{itemize}
\end{frame}


%-----------------------------------------------------------------------%
\begin{frame}[fragile]
\frametitle{Resources}
\begin{itemize}
\item A \emph{resource} is data that can be used and/or defined
  without being explicitly passed as a parameter
\item Similar to State and IO monads
\item Specified in procedure declaration, but not in call
\item Calls to procedures that use resources must be preceded with
  \texttt{!} to signify that they use some resources
\item Procedures can use as many resources as they want to declare
\item Resource can be declared as a name for several other resources
\item Useful for data that is widely used/modified in a module
\item I/O, command line arguments are resources visible at top level
  \begin{semiverbatim}
def hello(name) use io:
    !print("Hello, ")
    !println(name)
  \end{semiverbatim}
\end{itemize}
\end{frame}


%-----------------------------------------------------------------------%
\begin{frame}[fragile]
\frametitle{Tests}
\begin{itemize}
\item Some procedure calls, called \emph{tests}, can succeed or fail
\item Some modes of a procedure can be tests while others are not
\item Definition specifies that call can fail with \texttt{?} at left
  of signature
\item A test can also produce output: use it in place of a Maybe
\item \emph{e.g.,} \texttt{def ?cons(?head,?tail,list):} \ldots
\item A test with no outputs can be used as an expression:  it is
  reified into a \texttt{bool}
\item A call supplying an input where an output is expected is
  automatically a test that compares the output with the supplied input
(like Mercury's implied modes)
\end{itemize}
\end{frame}


%-----------------------------------------------------------------------%
\begin{frame}[fragile]
\frametitle{Tests as statements}
\begin{itemize}
\item A test can be used as a statement
\item Call must be preceded with \texttt{?}
\item Sequence of statements is a test if any of them are tests
\item Procedure is a test if its body is a test
\item All tests must be declared with \texttt{?} at left
\item Statement sequence fails if any of the tests fail
\item Like logic programming or the Maybe monad
\item If test fails, its effects are rolled back
\item I/O is not allowed in tests
\end{itemize}
\end{frame}


%-----------------------------------------------------------------------%
\begin{frame}[fragile]
\frametitle{Case statement and expression}
\begin{itemize}
\item \texttt{case} \emph{expr} \texttt{of} \\
\hspace*{1em} \emph{case1}\texttt{:} \emph{body1} \ldots \\
\hspace*{1em} \emph{case2}\texttt{:} \emph{body2} \ldots \\
\hspace*{1em} \ldots \\[2ex]
is equivalent to \\[2ex]
 \texttt{if} \emph{case1(expr)}\texttt{:} \emph{body1} \ldots \\
\hspace*{1em} \emph{case2(expr)}\texttt{:} \emph{body2} \ldots \\
\hspace*{1em} \ldots
\item Except that tests must be exhaustive: checked at compile-time
\item Can declare sets of exhaustive tests
\item \emph{E.g.:}\\
  \texttt{def ++(x,y) = case x of} \\
\hspace*{3em}
\begin{tabular}{rl}
\texttt{[]:} & \texttt{y}\\
\texttt{[?h|?t]:} & \texttt{[h|t++y]}\\
\end{tabular}
\end{itemize}
\end{frame}


%-----------------------------------------------------------------------%
\begin{frame}[fragile]
\frametitle{Generators}
\begin{itemize}
\item Generators are procedures that return any number of times
\item Like \texttt{nondet} predicates in Mercury; similar to the list monad
\item Generators are declared with a \texttt{*} before the signature
\item Generators use \texttt{generate} statements to specify
  multiple results
\item Each \texttt{generate} encloses multiple statements producing
  results from initial state
\item Results are produced in the specified order
\item Generators can call tests; if test fails, skip to next \texttt{generate}
\item \emph{E.g.,} \\
  \hspace*{2em}\texttt{def *in(?elt,list):} \\
  \hspace*{4em}\texttt{generate ??elt = head(list)} \\
  \hspace*{4em}\texttt{generate *?elt in tail(list)} \\

\end{itemize}
\end{frame}


%-----------------------------------------------------------------------%
\begin{frame}[fragile]
\frametitle{Generators}
\begin{itemize}
\item A procedure that calls a generator outside of a \texttt{for}
  construct is also a generator
\item A sequence of calls to generators generates the cross-product of
  their results
\item A test is like a generator restricted to at most one result
\item Generators can be used for logic programming-like coding
\item \texttt{for} statements allow iteration over generators,
  accumulating results
\item \texttt{do} loop can have multiple \texttt{for} statements to
  support lock-step iteration
\end{itemize}
\end{frame}


%-----------------------------------------------------------------------%
\begin{frame}[fragile]
\frametitle{Types}
\begin{itemize}
\item Type system is not designed yet; much work to be done
\item These are some goals:
  \begin{itemize}
  \item Strongly typed
  \item Type inference for local variables and formal parameters of
    private procedures
  \item Parametric polymorphism
  \item Declaration of algebraic types produces constructors,
    deconstructors, accessors, mutators
  \item Also possible to define all these as normal procedures, so one
    can directly implement types by defining their primitive
    operations
  \item \emph{E.g.,} can generate  constructors, deconstructors,
    accessors, mutators for C structs passed through foreign interface
  \end{itemize}
\end{itemize}
\end{frame}


%-----------------------------------------------------------------------%
\begin{frame}[fragile]
\frametitle{Types}
\begin{itemize}
\item Interface inheritance: unify types with type classes
  \begin{itemize}
  \item Abstract type $\equiv$ type class
  \item Allow a type $A$ to ``implement'' another type $B$, by
    defining all $B$'s primitive operations for type $A$
  \item Then an $A$ \emph{is-a} $B$: pass an $A$ where a $B$ is expected
  \item \emph{E.g.,} allow list processing functions to work on arrays
  by defining \texttt{car}, \texttt{cdr}, and \texttt{cons} for arrays
\end{itemize}
\item Implementation inheritance: declarative delegation
  \begin{itemize}
  \item For a specified set of procedures, 
    declare a function $f: a \to b$ to convert an $a$ argument to a $b$
  \item Allows passing an $a$ for any $b$ type parameter to these procedures
  \item Controlled coercion, or easy overloading
  \item Allows composition to substitute for (multiple) inheritance
  \item But no overriding
  \end{itemize}
\end{itemize}
\end{frame}


%-----------------------------------------------------------------------%
\begin{frame}[fragile]
\frametitle{Conclusion}
\begin{quotation}
\noindent
Simplicity does not precede complexity, but follows it.
\hspace*{3em}--- Alan Perlis
\end{quotation}
\vspace{3ex}
\begin{itemize}
\item Looking for ways to simplify Wybe while satisfying its goals
\item More to add, too:
  \begin{itemize}
  \item Higher order
  \item Declared non-strict parameters
  \item Lightweight parallelism (generators can support this)
  \item ``Identities'' and relations among them, to allow
    networks of objects to be navigated and mutated OO-style
  \end{itemize}

\end{itemize}
\end{frame}


% %-----------------------------------------------------------------------%
% \begin{frame}
% \frametitle{Type classes}
% \begin{itemize}
% \item 
% \end{itemize}
% \end{frame}



\end{document}

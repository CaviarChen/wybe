\documentclass[12pt]{beamer}
\mode<presentation>
\usetheme{Boadilla}
%\usepackage{beamerthemeGoettingen}
%\usepackage{beamerthemeDresden}

%%%%%%%%%%%%%%%%%%%%%%%%%%%%%%%%%%%%%%%%%%%%%%%%%%%%%%%%%%%%%%%%%%%%%%%%%%%%%%%%%%%%
% If true  then we have a longer  version for a 40 minutes talk
% If false then we have a shorter version for a 25 minutes talk
\newcommand{\longflag}{false}
\newcommand{\includelongversion}[2]{\ifthenelse{\equal{\longflag}{true}}{#1}{#2}}
%%%%%%%%%%%%%%%%%%%%%%%%%%%%%%%%%%%%%%%%%%%%%%%%%%%%%%%%%%%%%%%%%%%%%%%%%%%%%%%%%%%%

\usepackage{graphicx}
\usepackage{latexsym}
\usepackage{amssymb, amsmath, stmaryrd}
\usepackage{hhline}
\usepackage{subfig}
\usepackage{multirow}
\usepackage{tikz}
\usepackage{minibox}
\usepackage{ifthen}
\usepackage{xspace}
\usepackage{mathabx}
\usepackage{url}
\usepackage{framed,color}

\usetikzlibrary{arrows,shapes,calc,decorations.pathreplacing}
\definecolor{pinegreen}{rgb}{0,0.55,0.45}

% don't bother me with fulsome warning messages
\tolerance=100000

\renewcommand{\topfraction}{.9}
\renewcommand{\bottomfraction}{.9}
\renewcommand{\textfraction}{0.1}

\definecolor{shadecolor}{rgb}{0.8,0.9,1.0}

\newcommand{\sys}[1]{\textsc{#1}}
\newcommand{\re}[1]{\ensuremath{\mathtt{#1}}}
\newcommand{\svar}[1]{\ensuremath{\left \llbracket #1 \right \rrbracket}}

\newcommand{\ceil}[1]{\left\lceil#1\right\rceil}
\newcommand{\bexists}{\overline{\exists}}
\newcommand{\entails}{\vDash}
\newcommand{\essential}[1]{\left \llbracket #1 \right \rrbracket}
\newcommand{\fail}{\bf fail}
\newcommand{\lra}{\leftrightarrow}
\newcommand{\lu}{{L\overline{U}}}
\newcommand{\ra}{\rightarrow}
\newcommand{\fff}{{\cal F}}
\newcommand{\ttt}{{\cal T}}
\providecommand{\abs}[1]{\lvert#1\rvert}

\newcommand{\pset}[1]{{\cal P}(#1)}

%% Wrapping a tabular environment in a macro,
%% so it doesn't conflict with Tikz stuff.
\newcommand{\wtab}[1]{
  \begin{tabular}{c}
    #1
  \end{tabular}
}



%\xyoption{ps}
%\xyoption{color}
%\UseCrayolaColors
%\newcommand{\xyo}[1]{*+[o][F-]{#1}}
%\newcommand{\xyco}[2]{*+[o][F-:#1]{#2}}


%-----------------------------------------------------------------------%
\title[Introducing Wybe]{The Design and Implementation of Wybe}

%%%%%%%%%%%%%%%%%%%%%%%%%%%%%%%%%%%%%%%%%%%%%%%%%%%%%%%%%%%%%%%%
% Abstract:
%
% We present Wybe, a new language in the early stages of development.
% Wybe combines the best of declarative and imperative programming in
% a principled way.  It is intended to be easy to learn for beginner
% programmers, and also to scale up to large projects through good
% support for software engineering principles.
%%%%%%%%%%%%%%%%%%%%%%%%%%%%%%%%%%%%%%%%%%%%%%%%%%%%%%%%%%%%%%%%

\author[Peter Schachte]
{Peter Schachte}

\institute[Melbourne Uni]{The University of Melbourne \\ School of
  Computing and Information Systems}

\date{4 September 2019}

%% \ignore{
%%   \AtBeginSection[]
%%   {
%%      \begin{frame}<beamer>
%%      \frametitle{Outline}
%%      \tableofcontents[currentsection]
%%      \end{frame}
%%   }
%% }

\begin{document}

\frame{\titlepage}

%=======================================================================%
\section{Introduction}

%-----------------------------------------------------------------------%
\begin{frame}[fragile]
\frametitle{Motivation --- Wybe goals:}
\begin{itemize}
\item Be simple and easy to learn (but not necessarily orthodox)
\item Support declarative programming
\item Support imperative programming
\item Support best software engineering principles
\item Be efficient
\end{itemize}
\vspace{2ex}
\begin{quote}
  The lurking suspicion that something could be simplified is the
world's richest source of rewarding challenges. \\
\hspace*{1cm} --- Edsger Wybe Dijkstra
\end{quote}
\end{frame}

%=======================================================================%
\section{Interface Integrity}

%-----------------------------------------------------------------------%
\begin{frame}[fragile]
\frametitle{Interface Integrity}
\begin{itemize}
\item Major advantage of declarative languages:  
  no \emph{action at a distance} --- where change happens for no apparent reason
\item Caused by destructive update of aliased structures
\item Requires the programmer to have (and maintain!) a \emph{global}
  understanding of dynamic aliasing in program
\item Every function has two interfaces:
  \begin{itemize}
  \item \alert<2>{Apparent interface} appears in function declaration
    or call
  \item \alert<3>{Effective interface} is the information that passes into and out of function
  \end{itemize}
\item A function exhibits \alert<4>{interface integrity} if its apparent and
effective interfaces are identical --- no action at a distance

\end{itemize}

\end{frame}


%=======================================================================%
\section{Wybe Basics}

%-----------------------------------------------------------------------%
\begin{frame}[fragile]
\frametitle{Hello World}
\vspace*{-2ex}
\begin{itemize}
\item Like a scripting language, top-level statements in a module are executed
\item Top-level statements in a module are executed before top-level statements
  in any module that imports it
\item Hello World in Wybe:\vspace*{-1ex}
  \begin{minipage}{0.95\linewidth}
    \begin{block}{}
\begin{semiverbatim}
!println("Hello, World!")
\end{semiverbatim}
    \end{block}
  \end{minipage}
\item I'll explain the leading \texttt{!} a little later
\item Every Wybe source file is a module whose name is the file name with the
  \texttt{.wybe} extension removed.
\item Compiler is like make:  takes name of file to \emph{produce}\vspace*{-1ex}
  \begin{minipage}{0.95\linewidth}
    \begin{block}{}
\begin{semiverbatim}
\textbf{% }wybemk hello
\textbf{% }./hello
Hello, World!
\end{semiverbatim}
    \end{block}
  \end{minipage}
\end{itemize}
\end{frame}


%-----------------------------------------------------------------------%
\begin{frame}[fragile]
\frametitle{Information flow}
\begin{itemize}
\item Direction of information flow (\emph{mode}) is explicit
\item A bit like Ada's \texttt{in}, \texttt{out}, and \texttt{in out}
\item Unadorned variable name denotes variable value (call by value)
\item Question mark (\alert<2>{\texttt{?}}) in front of variable name indicates
  variable (re-)assignment (call by result)
\item Exclamation point (\alert<3>{\texttt{!}}) indicates both (call by value-result)
\item Which side of \texttt{=} code is on is irrelevant
\item Both of these increment \texttt{x}: \\[-1ex]
  \begin{minipage}{0.95\linewidth}
    \begin{block}{}
\begin{semiverbatim}
?x = x + 1
\end{semiverbatim}
    \end{block}
  \end{minipage} \\[-1ex]
  \begin{minipage}{0.95\linewidth}
    \begin{block}{}
\begin{semiverbatim}
x + 1 = ?x
\end{semiverbatim}
    \end{block}
  \end{minipage}
  \begin{minipage}[c]{0.4\linewidth}
  \end{minipage}
\item So does this: \\[-1ex]
  \begin{minipage}{0.95\linewidth}
    \begin{block}{}
\begin{semiverbatim}
incr(!x)
\end{semiverbatim}
    \end{block}
  \end{minipage}
\end{itemize}
\end{frame}


%-----------------------------------------------------------------------%
\begin{frame}[fragile]
\frametitle{Value semantics}
\begin{itemize}
\item Wybe has value semantics:
  \begin{itemize}
  \item Aliasing is not semantically significant
  \item Data structures cannot be mutated,
  \item But variables can be re-assigned
  \end{itemize}
\item \emph{Eg}, \texttt{!pos.x = new\_x} assigns \texttt{pos}
  a \emph{new} position structure whose \texttt{x} member is
  \texttt{new\_x} and
  whose other members are the same as the original \texttt{pos}
\item Aliases of \texttt{pos} are not affected
\item Compile-time garbage collection:  when analysis reveals structure is
  unaliased, compiler implements this with mutation
\item \\[-1ex]
  \begin{minipage}{0.15\linewidth}
    \begin{block}{}
\begin{semiverbatim}
pos.x
\end{semiverbatim}
    \end{block}
  \end{minipage}
  \quad is syntactic sugar for \quad
  \begin{minipage}{0.15\linewidth}
    \begin{block}{}
\begin{semiverbatim}
x(pos)
\end{semiverbatim}
    \end{block}
  \end{minipage}
\item \\[-1ex]
  \begin{minipage}{0.15\linewidth}
    \begin{block}{}
\begin{semiverbatim}
!pos.x
\end{semiverbatim}
    \end{block}
  \end{minipage}
  \quad is syntactic sugar for \quad
  \begin{minipage}{0.15\linewidth}
    \begin{block}{}
\begin{semiverbatim}
x(!pos)
\end{semiverbatim}
    \end{block}
  \end{minipage}
\end{itemize}
\end{frame}

\section{Procedures and Statements}

%-----------------------------------------------------------------------%
\begin{frame}[fragile]
\frametitle{Procedures}
\begin{itemize}
\item Same adornments are used in formal parameters \\[-1ex]
  \begin{minipage}{0.95\linewidth}
    \begin{block}{}
\begin{semiverbatim}
def foo(w, x, \alert{?}y, \alert{!}z) \{
    \ldots
\}
\end{semiverbatim}
    \end{block}
  \end{minipage}
\item Defines procedure with two
  inputs, one output, and one in-out parameter
\item Adornments in call must match definition (but see below\ldots)
\item This declares a private procedure; precede with \alert{\texttt{pub}}
  to export it
\item Body of a procedure definition is a sequence of
  statements
\item Procedure calls are statements
\item Plus there are a few built-in statement types, discussed below
\end{itemize}
\end{frame}


%-----------------------------------------------------------------------%
\begin{frame}[fragile]
\frametitle{If statement}
\begin{itemize}
\item Form of conditional statement: \\[-1ex]
  \begin{minipage}{0.95\linewidth}
    \begin{block}{}
\begin{semiverbatim}
if \{ \emph{test1} :: \emph{statements} \ldots
   | \emph{test2} :: \emph{statements} \ldots
   \ldots
\}
\end{semiverbatim}
    \end{block}
  \end{minipage}
\item Tests are tried in order; body of first to succeed is executed
\item If no test succeeds, no body is executed
% \item Boolean expression \emph{e} is de-reified into the test \emph{e}
%   \texttt{= true}
% \item \texttt{else} is a test that always succeeds
% \item Also an expression version of \texttt{if}, where
%   \emph{statements} are replaced with \emph{expressions}
% \item Tests of an \texttt{if} expression must be exhaustive
\item Variables bound in tests are scoped to the corresponding statements
\item There will be an \texttt{if} expression
\item There will be \texttt{case} expressions/statements with similar syntax
\end{itemize}
\end{frame}


%-----------------------------------------------------------------------%
\begin{frame}[fragile]
\frametitle{Loops}
\begin{itemize}
\item One modular looping construct: \\[-1ex]
  \begin{minipage}{0.95\linewidth}
    \begin{block}{}
\begin{semiverbatim}
    do \{}
        \emph{loop-statements} \ldots
    \}
\end{semiverbatim}
    \end{block}
  \end{minipage}
\item \emph{loop-statements} are any normal statements plus any
  special looping statements, including:
  \begin{itemize}
  \item \alert{\texttt{while} \emph{test}} and
    \alert{\texttt{until} \emph{test}}
    --- like conditional \texttt{break}
  \item \alert{\texttt{when} \emph{test}} and
    \alert{\texttt{unless} \emph{test}}
  --- like conditional \texttt{continue}
  \end{itemize}
\item Include as many of these as you like wherever you like \\[-1ex]
  \begin{minipage}{0.95\linewidth}
    \begin{block}{}
\begin{semiverbatim}
    do \{!print(prompt)
        !readln(?answer)
        until answer in ["y","n"]
        !println("Please answer 'y' or 'n'.")
    \}
\end{semiverbatim}
    \end{block}
  \end{minipage}
\end{itemize}
\end{frame}


%-----------------------------------------------------------------------%
\begin{frame}[fragile]
\frametitle{Expressions and Functions}
\begin{itemize}
\item Procedure call arguments are expressions
\item Expressions are just procedure calls with last argument omitted
\item The value of such an expression is the value that would be assigned
  to its omitted argument
\item \emph{E.g.,} \alert{\texttt{bar(x,y)}} as an expression means call
  \alert{\texttt{bar(x,y,?temp)}} and use \alert{\texttt{temp}} as the value of
  \alert{\texttt{bar(x,y)}}
\item \\[-1ex]
  \begin{minipage}{0.35\linewidth}
    \begin{block}{}
\begin{semiverbatim}
foo(bar(x,y),?z)
\end{semiverbatim}
    \end{block}
  \end{minipage}
  \quad means \quad
  \begin{minipage}{0.3\linewidth}
    \begin{block}{}
\begin{semiverbatim}
bar(x,y,?temp)
foo(temp,?z)
\end{semiverbatim}
    \end{block}
  \end{minipage}
  \begin{minipage}[c]{0.4\linewidth}
  \end{minipage}
\item \\[-1ex]
  \begin{minipage}{0.45\linewidth}
    \begin{block}{}
\begin{semiverbatim}
def foo(x) = bar(x,x)
\end{semiverbatim}
    \end{block}
  \end{minipage}
\quad is syntactic sugar for \\[-.5ex]
  \begin{minipage}{0.95\linewidth}
    \begin{block}{}
\begin{semiverbatim}
def foo(x,?result) \{ bar(x,x,?result) \}
\end{semiverbatim}
    \end{block}
  \end{minipage}
\item Either definition allows \alert{\texttt{foo}} to be used with
  either syntax
% \item A few built-in expressions like \texttt{let \{} \emph{stmts}
%   \texttt{\} in} \emph{expr} and \emph{expr} \texttt{where \{} \emph{stmts}
%   \texttt{\}}
% \item \texttt{a.b(c,\ldots)} is syntactic sugar for
%   \texttt{b(a,c,\ldots)}, to allow dot notation for member access and update
\end{itemize}
\end{frame}


%-----------------------------------------------------------------------%
\begin{frame}[fragile]
\frametitle{Reversibility}
\begin{itemize}
\item Procedures can be overloaded based on mode:
  \begin{itemize}
  \item \\[-1ex]
    \begin{minipage}{0.45\linewidth}
      \begin{block}{}
\begin{semiverbatim}
position(x, y, ?pos)
\end{semiverbatim}
      \end{block}
    \end{minipage} \quad constructs a position
  \item \\[-1ex]
    \begin{minipage}{0.45\linewidth}
      \begin{block}{}
\begin{semiverbatim}
position(?x, ?y, pos)
\end{semiverbatim}
      \end{block}
    \end{minipage} \quad deconstructs
  \end{itemize}
\item This is nicer as an expression (pattern matching):
  \begin{itemize}
  \item \\[-1ex]
    \begin{minipage}{0.45\linewidth}
      \begin{block}{}
\begin{semiverbatim}
?pos = position(x, y)
\end{semiverbatim}
      \end{block}
    \end{minipage} \quad constructs a position
  \item \\[-1ex]
    \begin{minipage}{0.45\linewidth}
      \begin{block}{}
\begin{semiverbatim}
pos = position(?x, ?y)
\end{semiverbatim}
      \end{block}
    \end{minipage} \quad deconstructs
  \item Use deconstructor expression for any output argument
    (pattern match anywhere)
  \end{itemize}
\end{itemize}
\end{frame}

\section{Modules, Types, and Resources}

%-----------------------------------------------------------------------%
\begin{frame}[fragile]
\frametitle{Modules}
\begin{itemize}
\item Each source file is a module
\item A directory of source files is also a module
\item A (sub)-module can be declared within a module: \\[-1ex]
  \begin{minipage}{0.95\linewidth}
    \begin{block}{}
\begin{semiverbatim}
module \emph{name} \{
    \emph{items}
\}
\end{semiverbatim}
    \end{block}
  \end{minipage}
\item Items in a submodule can be anything that can be in a module
\item Submodule has access to everything in parent module
\item Parent module can only access public contents of submodule
\item Can import public contents of a module with \\[-1ex]
  \begin{minipage}{0.95\linewidth}
    \begin{block}{}
\begin{semiverbatim}
use \emph{module}
\end{semiverbatim}
    \end{block}
  \end{minipage}
\end{itemize}
\end{frame}


%-----------------------------------------------------------------------%
\begin{frame}[fragile]
\frametitle{Types}
\begin{itemize}
\item A Wybe type is just a (sub)module that can be used as a type \\[-1ex]
  \begin{minipage}{0.95\linewidth}
    \begin{block}{}
\begin{semiverbatim}
type \emph{name} \{
    \emph{constructor} | \emph{constructor} | \ldots
    \emph{items}
\}
\end{semiverbatim}
    \end{block}
  \end{minipage}
\item each constructor looks like function declaration without a body
\item \emph{items} are other module items (procedures, types, \emph{etc}.)
\item Automatically generates code for constructors, deconstructors
\item Automatically generates accessor and mutator for each member
\item These are all ordinary Wybe code; you could write them yourself
\item Mayer's Uniform Access Principle: can change representation without
  changing type users
\end{itemize}
\end{frame}


%-----------------------------------------------------------------------%
\begin{frame}[fragile]
\frametitle{Resources}
\begin{itemize}
\item Resources provide convenient, non-positional argument passing
\item A \emph{resource} is declared as a name and type: \\[-1ex]
  \begin{minipage}{0.95\linewidth}
    \begin{block}{}
\begin{semiverbatim}
resource \emph{name} : \emph{type}
\end{semiverbatim}
    \end{block}
  \end{minipage}
\item A procedure indicates resources it uses with a \texttt{use}
  clause:
\\[-1ex]
  \begin{minipage}{0.95\linewidth}
    \begin{block}{}
\begin{semiverbatim}
def \emph{name}(\ldots) \alert{use \emph{resources}} \{ \ldots
\end{semiverbatim}
    \end{block}
  \end{minipage} \\[1ex]
\item \emph{resources} is comma-separated set of resource names,
  optionally prefixed with \texttt{?} or \texttt{!} to indicate flow
\item Resources become variables in procedure body, like parameters
\item Procedures calls do not explicitly list resources, they come from caller
  variables
\item Resourceful calls must be preceded with \texttt{!}
\item Useful for data that is widely used in a module
\item Predefined resources:  \texttt{io}, command line, \texttt{exit\_code}
\end{itemize}
\end{frame}


%-----------------------------------------------------------------------%
\begin{frame}[fragile]
\frametitle{Tests}
\begin{itemize}
\item Tests in conditionals can be Boolean-valued expressions
\item They can also be \alert{\emph{test statements}}: statements that can fail
\item For example,\quad
    \begin{minipage}{0.35\linewidth}
      \begin{block}{}
\begin{semiverbatim}
lst = cons(?h,?t)
\end{semiverbatim}
      \end{block}
    \end{minipage}\quad
 fails if \texttt{lst} is \texttt{nil}
\item So do\quad
    \begin{minipage}{0.2\linewidth}
      \begin{block}{}
\begin{semiverbatim}
head(lst)
\end{semiverbatim}
      \end{block}
    \end{minipage}\quad
 \texttt{} and\quad
    \begin{minipage}{0.2\linewidth}
      \begin{block}{}
\begin{semiverbatim}
tail(lst)
\end{semiverbatim}
      \end{block}
    \end{minipage}
\item Test outputs can only be used when test has succeeded
\item Can only use a test as a condition or in definition of a test
\item Some modes of a procedure can be tests while others are not
\item Declare with \texttt{test} keyword:
    \begin{minipage}{0.5\linewidth}
      \begin{block}{}
\begin{semiverbatim}
def \alert{test} \textit{name}(\ldots) \ldots
\end{semiverbatim}
      \end{block}
    \end{minipage}\quad


\item Types with more than one constructor automatically do this for
  deconstructors, accessors, and mutators
\item A call supplying an input where an output is expected is
  automatically a test that compares output with supplied input
\end{itemize}
\end{frame}


\section{Implementation}

%-----------------------------------------------------------------------%
\begin{frame}[fragile]
\frametitle{Intermediate Representation}
\begin{itemize}
\item Source code is compiled to low-level Prolog-like deterministic
  strongly moded declarative language (LPVM)
\item Actually quite close to machine language
\item Instructions are low level arithmetic, comparisons, memory allocation,
  access, mutation, and procedure calls
\item Code is tree-structured, with exclusive, exhaustive branching on variable
  value (deterministic)
\item Loops are transformed into recursion
\item Branches do not merge
\item Code after conditionals and loops is turned into a separate procedure
  called at the end of each branch and each loop exit
\item Each variable only assigned once along any branch
\item Heavy use of inlining to avoid unnecessary calls
\end{itemize}
\end{frame}


%-----------------------------------------------------------------------%
\begin{frame}[fragile]
  \frametitle{LPVM example (types omitted for brevity)}
  \begin{minipage}{.35\linewidth}
\centerline{\textbf{Wybe source}}
\begin{semiverbatim}
def fact(n, ?f) \{
    ?f = 1
    do \{while n > 1
        ?f = f * n
        ?n = n - 1
    \}
\}
\end{semiverbatim}
  \end{minipage}
\qquad
  \begin{minipage}{.55\linewidth}
\centerline{\textbf{LPVM code}}
\begin{semiverbatim}
fact(n, ?f) :-
    gen1(n, 1, ?f).

gen1(n, f, ?f#2):
    llvm icmp sgt(n, 1, ?tmp3)
    ( tmp3 = 0
    ->  llvm move(f, ?f#2)
    ; tmp3 = 1
    ->  llvm mul(n, f, ?tmp0)
        llvm sub(n, 1, ?tmp1)
        gen1(tmp1, tmp0, ?f#2)
    ).
\end{semiverbatim}
  \end{minipage}
\end{frame}


%-----------------------------------------------------------------------%
\begin{frame}[fragile]
\frametitle{LPVM advantages}
\begin{itemize}
\item Clean simple semantics:
  \begin{itemize}
  \item each path is just a conjunction
  \item each branch point is just a disjunction
  \end{itemize}
\item Easy reordering:  just keep variable assignment before first use
\item Dead code elimination: omit instructions with unused outputs
\item Inlining:  just whack in procedure body with renamed variables
\item Easy analysis:
  \begin{itemize}
  \item Single assignment on each branch
  \item No Phi nodes
  \item commutative conjunction:
    forward and backward analysis are equally easy
  \item Each branch includes its path conditions (though reification of
    conditions complicates analysis)
  \end{itemize}
\end{itemize}
\end{frame}


%-----------------------------------------------------------------------%
\begin{frame}[fragile]
\frametitle{Data structure implementation}
\begin{itemize}
\item Low-level instructions used for data structures:
  \begin{itemize}
  \item \texttt{alloc(}\emph{size}, \emph{?addr}\texttt{)} \\
    allocate a block of \emph{size} byes of memory
  \item \texttt{access(}\emph{addr}, \emph{offset}, \emph{?value}\texttt{)} \\
    access the value at \emph{offset} bytes beyond \emph{addr} in memory
  \item \texttt{mutate(}\emph{addr}, \emph{?newaddr}, \emph{size},
    \emph{offset}, \emph{destr}, \emph{newvalue}\texttt{)} \\
    if \emph{destr} is 1, \emph{newaddr = addr}, and destructively store
    \emph{newvalue} in memory at the address \emph{offset} bytes beyond
    \emph{addr}; if \emph{destr} is 0, then allocate \emph{size} bytes at
    \emph{newaddr} and copy from \emph{addr} to \emph{newaddr} before storing
  \end{itemize}
\item Aliasing and liveness analysis determine when a mutated structure is not
  aliased, and sets the \emph{destr} arg to 1.
\item For this to work inter-procedurally, need multiple specialisation
\item Currently using Boehm conservative garbage collector
\end{itemize}
\end{frame}


%-----------------------------------------------------------------------%
\begin{frame}[fragile]
\frametitle{Compilation process}
\begin{itemize}
\item Compilation bottom-up by SCCs in module dependency graph
\item If target is older than source file:
\begin{enumerate}
\item Read source file; compile code as far as possible without
  considering dependencies
\item Recursively load dependencies, compiling if necessary,
  producing a single SCC in module dependency graph
\item For each SCC in type dependency graph, generate code for types
\item Type and mode check procedures
\item Flatten procedure definitions and generate LPVM
\item Analyse and optimise each call graph SCC bottom-up:
  constant folding,
  common subexpression elimination, dead code elimination, unneeded
  argument elimination, liveness analysis, inlining,
  aliasing analysis, compile-time garbage collection
\item LLVM code is generated, object files are written
\end{enumerate}
\item Executables are generated
\end{itemize}
\end{frame}

\section{Conclusions}

%-----------------------------------------------------------------------%
\begin{frame}[fragile]
\frametitle{Future work}
\begin{itemize}
\item Higher order procedures/functions
\item Parametric polymorphism
\item User-defined control structures through declared lazy parameters
\item Tooling:  incremental compilation, debugging, wybedoc, etc
\item Support downward analysis by using info from past compilations
\item Generators:  produce multiple outputs one at a time
\item Open types: a type can add constructors to another type
  \begin{itemize}
  \item extended type can be used where the extending type is required
  \item extending type can be used where the extended type is required if the
  extra constructors have been excluded
  \item For example, \texttt{list(t)} extends \texttt{nonempty\_list(t)}
  \end{itemize}
\item Interface inheritance through type re-implementation
  \begin{itemize}
  \item Declare type implements another concrete or abstract type
  \item Can use that type wherever you could use the other one
  \item \emph{Eg}, use \texttt{int} as \texttt{number} or \texttt{array} as
    \texttt{list}
  \end{itemize}
\item Improved automatic memory management
\end{itemize}
\end{frame}


% %-----------------------------------------------------------------------%
% \begin{frame}[fragile]
% \frametitle{Types}
% \begin{itemize}
% \item Type system is not designed yet; much work to be done
% \item These are some goals:
%   \begin{itemize}
%   \item Strongly typed
%   \item Type inference for local variables and formal parameters of
%     private procedures
%   \item Parametric polymorphism
%   \item Declaration of algebraic types produces constructors,
%     deconstructors, accessors, mutators
%   \item Also possible to define all these as normal procedures, so one
%     can directly implement types by defining their primitive
%     operations
%   \item \emph{E.g.,} can generate  constructors, deconstructors,
%     accessors, mutators for C structs passed through foreign interface
%   \end{itemize}
% \end{itemize}
% \end{frame}


% %-----------------------------------------------------------------------%
% \begin{frame}[fragile]
% \frametitle{Types}
% \begin{itemize}
% \item Interface inheritance: unify types with type classes
%   \begin{itemize}
%   \item Abstract type $\equiv$ type class
%   \item Allow a type $A$ to ``implement'' another type $B$, by
%     defining all $B$'s primitive operations for type $A$
%   \item Then an $A$ \emph{is-a} $B$: pass an $A$ where a $B$ is expected
%   \item \emph{E.g.,} allow list processing functions to work on arrays
%   by defining \texttt{car}, \texttt{cdr}, and \texttt{cons} for arrays
% \end{itemize}
% \item Implementation inheritance: declarative delegation
%   \begin{itemize}
%   \item For a specified set of procedures, 
%     declare a function $f: a \to b$ to convert an $a$ argument to a $b$
%   \item Allows passing an $a$ for any $b$ type parameter to these procedures
%   \item Controlled coercion, or easy overloading
%   \item Allows composition to substitute for (multiple) inheritance
%   \item But no overriding
%   \end{itemize}
% \end{itemize}
% \end{frame}


%-----------------------------------------------------------------------%
\begin{frame}[fragile]
\frametitle{Conclusion}
% \begin{quotation}
% \noindent
% Simplicity does not precede complexity, but follows it.
% \hspace*{3em}--- Alan Perlis
% \end{quotation}

% \vspace{2ex}
% Always looking for ways to simplify Wybe while satisfying its goals

\begin{center}
\large
\vspace{-8ex}
Always interested in collaborative projects

\vspace{8ex}
Questions?
\end{center}
\end{frame}

% %-----------------------------------------------------------------------%
% \begin{frame}
% \frametitle{Type classes}
% \begin{itemize}
% \item 
% \end{itemize}
% \end{frame}



\end{document}

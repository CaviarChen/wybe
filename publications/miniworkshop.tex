\documentclass[12pt]{beamer}
\mode<presentation>
\usetheme{Boadilla}
%\usepackage{beamerthemeGoettingen}
%\usepackage{beamerthemeDresden}

%%%%%%%%%%%%%%%%%%%%%%%%%%%%%%%%%%%%%%%%%%%%%%%%%%%%%%%%%%%%%%%%%%%%%%%%%%%%%%%%%%%%
% If true  then we have a longer  version for a 40 minutes talk
% If false then we have a shorter version for a 25 minutes talk
\newcommand{\longflag}{false}       
\newcommand{\includelongversion}[2]{\ifthenelse{\equal{\longflag}{true}}{#1}{#2}}
%%%%%%%%%%%%%%%%%%%%%%%%%%%%%%%%%%%%%%%%%%%%%%%%%%%%%%%%%%%%%%%%%%%%%%%%%%%%%%%%%%%%

\usepackage{graphicx}
\usepackage{latexsym}
\usepackage{amssymb, amsmath, stmaryrd}
\usepackage{hhline}
\usepackage{subfig}
\usepackage{multirow}
\usepackage{tikz}
\usepackage{minibox}
\usepackage{ifthen}
\usepackage{xspace}
\usepackage{mathabx}
\usepackage{url}

\usetikzlibrary{arrows,shapes,calc,decorations.pathreplacing}
\definecolor{pinegreen}{rgb}{0,0.55,0.45}

% don't bother me with fulsome warning messages
\tolerance=100000

\renewcommand{\topfraction}{.9}
\renewcommand{\bottomfraction}{.9}
\renewcommand{\textfraction}{0.1}



\newcommand{\sys}[1]{\textsc{#1}}
\newcommand{\re}[1]{\ensuremath{\mathtt{#1}}}
\newcommand{\svar}[1]{\ensuremath{\left \llbracket #1 \right \rrbracket}}

\DeclareMathOperator{\pl}{\!+}

\newcommand{\ceil}[1]{\left\lceil#1\right\rceil}
\newcommand{\bexists}{\overline{\exists}}
\newcommand{\entails}{\vDash}
\newcommand{\essential}[1]{\left \llbracket #1 \right \rrbracket}
\newcommand{\fail}{\bf fail}
\newcommand{\lra}{\leftrightarrow}
\newcommand{\lu}{{L\overline{U}}}
\newcommand{\ra}{\rightarrow}
\newcommand{\fff}{{\cal F}}
\newcommand{\ttt}{{\cal T}}
\providecommand{\abs}[1]{\lvert#1\rvert}

\newcommand{\pset}[1]{{\cal P}(#1)}

%% Wrapping a tabular environment in a macro,
%% so it doesn't conflict with Tikz stuff.
\newcommand{\wtab}[1]{
  \begin{tabular}{c}
    #1
  \end{tabular}
}


%\xyoption{ps}
%\xyoption{color}
%\UseCrayolaColors
%\newcommand{\xyo}[1]{*+[o][F-]{#1}}
%\newcommand{\xyco}[2]{*+[o][F-:#1]{#2}}


%-----------------------------------------------------------------------%
\title[Introducing Wybe]{Introducing Wybe --- a language for everyone}

%%%%%%%%%%%%%%%%%%%%%%%%%%%%%%%%%%%%%%%%%%%%%%%%%%%%%%%%%%%%%%%%
% Abstract:
%
% We present Wybe, a new language in the early stages of development.
% Wybe combines the best of declarative and imperative programming in
% a principled way.  It is intended to be easy to learn for beginner
% programmers, and also to scale up to large projects through good
% support for software engineering principles.
%%%%%%%%%%%%%%%%%%%%%%%%%%%%%%%%%%%%%%%%%%%%%%%%%%%%%%%%%%%%%%%%

\author[Peter Schachte]
{\textbf{Peter Schachte}  \\
  \small joint work with Matthew Giuca}

\institute[Melbourne]{The University of Melbourne \\ Department of
  Computing and Information Systems}

\date{4 December 2013}

%% \ignore{
%%   \AtBeginSection[]
%%   {
%%      \begin{frame}<beamer>
%%      \frametitle{Outline}
%%      \tableofcontents[currentsection]
%%      \end{frame}
%%   }
%% }

\begin{document}

\frame{\titlepage}

%=======================================================================%
\section{Introduction}

%-----------------------------------------------------------------------%
\begin{frame}
\frametitle{Motivation}
\begin{itemize}
\item Many students have difficulty learning to program
\item Python is simple and easy to learn, but:
  \begin{itemize}
  \item It's not efficient enough for many uses
  \item Lack of static type checking hampers its use in large projects
  \end{itemize}
\item Java is efficient and scales well, but:
  \begin{itemize}
  \item It is rather complex
  \item It has numerous pitfalls
  \end{itemize}
\item Haskell is efficient and fairly simple, but:
  \begin{itemize}
  \item Students have trouble with some of its concepts
  \item Many students don't find it intuitive
  \end{itemize}
\item Need a language to take us from learning through to practice
\end{itemize}
\end{frame}

%-----------------------------------------------------------------------%
\begin{frame}
\frametitle{Student issues with Java}
\begin{itemize}
\item Aliasing and its dangers, defensive copying, immutability
\item Deep \emph{vs.} shallow copying and equality
\item Difference between primitive, object, and array types
\item Static variables/methods and static classes
\item Privacy, inheritance, late binding, static members
\item Packages and build systems
\end{itemize}

But they can write non-trivial programs --- build-and-fix works

\end{frame}


%-----------------------------------------------------------------------%
\begin{frame}
\frametitle{Student issues with Haskell and Mercury}
\begin{itemize}
\item Recursion
\item Lack of destructive update
\item Types of partially applied functions
\item Monads
\item Nondeterminism
\item Some of the error messages
\item Lack of a good IDE, debugger, REPL
\item The numeric type classes (a bit)
\item How (or why?) to take advantage of algebraic types
\end{itemize}

\end{frame}


%=======================================================================%
\section{Interface Integrity}

%-----------------------------------------------------------------------%
\begin{frame}
\frametitle{Action at a distance}
\begin{itemize}
\item Several of the problems students have with Java stem from
  \emph{action at a distance} --- change happens for no apparent reason
\item The problem:  destructive update of aliased structures
\item This is also a practical problem for software engineers
\item Must have a mental model of computer memory
\item Must have a \emph{global} understanding of aliasing

\end{itemize}

\end{frame}


%-----------------------------------------------------------------------%
\begin{frame}
\frametitle{Software Engineering}
\begin{itemize}
\item For code to be \emph{maintainable}, callers and callees should be
  able to develop and maintain their code independently
\item A \emph{local} understanding of each unit of code must be sufficient
\item This requires a formal \emph{interface} between callers and callees
\item But there are really two interfaces:
  \begin{itemize}
  \item The \textbf{apparent interface} is what appears in declaration
    or call syntax
  \item The \textbf{effective interface} between callers and
    callees is the information that passes between them
  \end{itemize}
\end{itemize}

\end{frame}


%-----------------------------------------------------------------------%
\begin{frame}
\frametitle{Interface integrity}

A function exhibits \textbf{Interface integrity} if its apparent and
effective interfaces coincide.

\begin{itemize}
\item This rules out:
  \begin{itemize}
  \item Destructive update of aliased structures, since this would allow
    information flow not reflected in the apparent interface
  \item Global variables, which would allow information flow from
    assignment to reference not reflected in the apparent interfaces
  \item I/O (information flow into/out of the environment) without
    indication in the apparent interfaces
  \item Unchecked exceptions
  \end{itemize}
\item This does not rule out:
  \begin{itemize}
  \item Variable reassignment
  \item Looping constructs
  \end{itemize}
\item Do what you like \emph{inside} a function --- as long as it's not
  observable \emph{outside}
\end{itemize}

\end{frame}


%=======================================================================%
\section{Wybe Basics}

%-----------------------------------------------------------------------%
\begin{frame}
\frametitle{Wybe basics}
\begin{quotation}
Simplicity is prerequisite for reliability. \\
\hspace*{3em}--- Edsger Wybe Dijkstra
\end{quotation}

\begin{itemize}
\item Wybe is designed to:
  \begin{itemize}
  \item Enforce interface integrity
  \item Be easy to learn
  \item Scale to large applications
  \item Allow efficient implementation
  \item Support both functional and imperative programming
  \end{itemize}
\item Wybe is in the early design stages
\item The syntax is not settled yet; take the following as an early conception
\end{itemize}
\end{frame}


%-----------------------------------------------------------------------%
\begin{frame}
\frametitle{Hello World}
\begin{itemize}
\item Comments introduced by hash (\texttt{\#})
\item Hello World in Wybe: \\[3ex]
\begin{alltt}
    \hspace*{5em}\texttt{\#!/usr/bin/env wybe}\\
    \hspace*{5em}\texttt{!println("Hello, World")}
\end{alltt} \\[3ex]
\item Like a scripting language, top-level statements become main
\item I'll explain the \texttt{!} later
\end{itemize}
\end{frame}


%-----------------------------------------------------------------------%
\begin{frame}
\frametitle{Information flow}
\begin{itemize}
\item Direction of information flow (\emph{mode}) is explicit
\item A bit like Ada's \texttt{in}, \texttt{out}, and \texttt{in out}
\item Unadorned variable name denotes variable value (call by value)
\item Question mark (\texttt{?}) in front of variable name indicates
  variable assignment (call by result)
\item Exclamation point (\texttt{!}) indicates both (call by value-result)
\item \texttt{?x = x + 1} or \texttt{x + 1 = ?x} increments x
\item so does \texttt{incr(!x)}
\item \texttt{=} and \texttt{incr} are library procedures
\end{itemize}
\end{frame}


%-----------------------------------------------------------------------%
\begin{frame}
\frametitle{Procedures}
\begin{itemize}
\item Same adornments are used in formal parameters
\item \texttt{def foo(w, x, ?y, !z):} \ldots\ defines procedure with two
  inputs, one output, and one in-out parameter
\item Adornments in call must match definition (but see below\ldots)
\item Body of a procedure definition is a sequence of
  procedure calls
\end{itemize}
\end{frame}


%-----------------------------------------------------------------------%
\begin{frame}
\frametitle{Expressions}
\begin{itemize}
\item Procedure arguments can be expressions
\item A function call is just a procedure call with the final argument omitted
\item Call stands for value of final argument
\item \texttt{foo(bar(x,y),?z)} \quad means \quad
  \begin{minipage}[c]{0.4\linewidth}
  \texttt{bar(x,y,?temp)} \\
  \texttt{foo(temp,?z)}
  \end{minipage}
\item \texttt{def foo(x) = bar(x,x)} is syntactic sugar for \\
\texttt{def foo(x,?result): bar(x,x,?result)}
\item Can use \texttt{foo} with either syntax regardless of which
  definition was used
\item \texttt{a.b(c,\ldots)} is syntactic sugar for
  \texttt{b(a,c,\ldots)}, to allow dot notation for member access and update
\end{itemize}
\end{frame}


%-----------------------------------------------------------------------%
\begin{frame}
\frametitle{Reversibility}
\begin{itemize}
\item Procedures can be overloaded based on mode
\item \texttt{cons(head,tail,?list)}  constructs \\
\texttt{cons(?head,?tail,list)}  deconstructs \\
\item Expressions can be \emph{outputs} (patterns) as well as inputs
\item Mode of final parameter must be determined by the others
\item Expression \texttt{cons(h,t)} constructs list \\
Expression \texttt{cons(?h,?t)} deconstructs
\item head(tail(!x), y) replaces head of tail of x with y
\item head(tail(!x)) = y is exactly the same
\item head(tail(!x), y) \quad transforms to \quad
  \begin{minipage}[c]{0.4\linewidth}
  \texttt{tail(x,?temp)} \\
  \texttt{head(!temp,y)} \\
  \texttt{tail(!x, temp)}
  \end{minipage}
% \item \texttt{def \_(x):} defines \texttt{\_} as a ``don't
%   care'' output
\end{itemize}
\end{frame}


%-----------------------------------------------------------------------%
\begin{frame}
\frametitle{Value semantics}
\begin{itemize}
\item Wybe has value semantics:  aliasing is not semantically significant
\item \texttt{head(!list,val)} does \emph{not} mean \texttt{RPLACA}
\item Equivalent to \texttt{?list = cons(val,tail(list))}
\item Gives the feeling of changing values without action at a distance
\item May be surprising to experienced programmers\ldots
\item But (hopefully) natural for beginners
\item Compile-time garbage collection:  when safe, compiler transforms
  this (back) into destructive modification
\item Can this be made predictable enough for programmers
  to have a good performance model and to write efficient code?
\item Can this make declarative programming with arrays \texttt{etc.} practical?
\end{itemize}
\end{frame}


%-----------------------------------------------------------------------%
\begin{frame}
\frametitle{Tests}
\begin{itemize}
\item Some procedure calls, called \emph{tests}, can succeed or fail
\item Depends on mode
\item Definition specifies that call can fail with \texttt{?} at end
  of signature
\item Without \texttt{?} it cannot fail
\item \emph{e.g.,} \texttt{def cons(?head,?tail,list)\textbf{?}:} \ldots
\item Sequence of tests and normal calls fails if any of the tests fail
\item Like logic programming or the Maybe monad
\item If body can fail, procedure must be declared with \texttt{?}
  before \texttt{:}
\item Test can be used as an expression:  it is reified into a bool
\item Probably must give up currying and overloading based on arity
\end{itemize}
\end{frame}


%-----------------------------------------------------------------------%
\begin{frame}
\frametitle{If statement}
\begin{itemize}
\item Tests can be used in \texttt{if} and \texttt{case} statements
\item \texttt{if} \emph{test1}\texttt{:} \emph{statements} \ldots \\
\hspace*{1em} \emph{test2}\texttt{:} \emph{statements} \ldots \\
\hspace*{1em} \ldots \\
\item Boolean expression \emph{e} is de-reified into the test \emph{e}
  \texttt{= true}
\item \texttt{def else = true}
% \item Variables bound in tests are scoped to the corresponding statements
\item An if statement not guaranteed to succeed is itself a test
\item How smart can we be about what's guaranteed to succeed?
\item Like Mercury's switch detection
\item Also allow declaration of disjunctive tautologies, \emph{e.g.}:\\
  \centerline{\texttt{x < y or x = y or x > y}}
\end{itemize}
\end{frame}


%-----------------------------------------------------------------------%
\begin{frame}
\frametitle{Case statement}
\begin{itemize}
\item \texttt{case} \emph{expr} \texttt{of} \\
\hspace*{1em} \emph{case1}\texttt{:} \emph{statements1} \ldots \\
\hspace*{1em} \emph{case2}\texttt{:} \emph{statements2} \ldots \\
\hspace*{1em} \ldots \\[2ex]
is equivalent to \\[2ex]
 \texttt{if} \emph{case1(expr)}\texttt{:} \emph{statements1} \ldots \\
\hspace*{1em} \emph{case2(expr)}\texttt{:} \emph{statements2} \ldots \\
\hspace*{1em} \ldots
\item \emph{E.g.:}\\
  \texttt{def ++(x,y) = case x of} \\
\hspace*{3em}
\begin{tabular}{rl}
\texttt{[]:} & \texttt{y}\\
\texttt{[?h|?t]:} & \texttt{[h|t++y]}\\
\end{tabular}
\end{itemize}
\end{frame}


%-----------------------------------------------------------------------%
\begin{frame}
\frametitle{Loops}
\begin{itemize}
\item One modular looping construct: \texttt{do} \emph{loop-statements} \ldots
\item \emph{loop-statements} are any normal statements plus any
  special looping statements, including:
  \begin{itemize}
  \item \texttt{while} \emph{test} and \texttt{until} \emph{test}
    \begin{itemize}
    \item like conditional \texttt{break}
    \end{itemize}
  \item \texttt{when} \emph{test} and \texttt{unless} \emph{test}
    \begin{itemize}
    \item like conditional \texttt{continue}
    \end{itemize}
  \item \texttt{for} \emph{generator}
  \end{itemize}
\item Include as many of these constructs as you like in the loop,
  wherever you like, \emph{e.g.:} \\
\begin{alltt}
\hspace*{2em}do  !print(prompt) \\
\hspace*{3em}    !readln(?answer) \\
\hspace*{3em}    until answer in ["y","n"] \\
\hspace*{3em}    !println("Please answer 'y' or 'n'.") \\
\end{alltt}
\end{itemize}
\end{frame}


%-----------------------------------------------------------------------%
\begin{frame}
\frametitle{Generators}
\begin{itemize}
\item Generators are procedures that return a multiple times
\item Like \texttt{multi} predicates in Mercury; similar to the list monad
\item A procedure that calls a generator outside of a \texttt{for}
  construct is itself a generator
\item Generators call \texttt{yield} to return current values of outputs
\item Generators can call other generators, and produce all
  combinations of outputs
\item Generators can call tests; if test fails, skip that result
\item Generators are declared with a \texttt{*} after the signature
\item \emph{E.g.,} \\
  \hspace*{2em}\texttt{def in(?elt,list)*:} \\
  \hspace*{4em}\texttt{?elt = list.head} \\
  \hspace*{4em}\texttt{yield} \\
  \hspace*{4em}\texttt{?elt in list.tail} \\

\end{itemize}
\end{frame}


%-----------------------------------------------------------------------%
\begin{frame}
\frametitle{Types}
\begin{itemize}
\item Type system is not designed yet; much work to be done
\item These are some goals
\item Strongly typed
\item Type inference for local variables and formal parameters of
  private procedures
\item Declaration of algebraic types produces
  constructors, deconstructors, accessors, mutators
\item Also possible to define all these as normal procedures, so one can
  directly implement types by defining their primitive operations
\end{itemize}
\end{frame}


%-----------------------------------------------------------------------%
\begin{frame}
\frametitle{Types}
\begin{itemize}
\item Unify types with type classes
  \begin{itemize}
  \item Abstract type $\equiv$ type class
  \item Allow a type $A$ to ``implement'' another type $B$, by
    defining all $B$'s primitive operation for type $A$
  \item Then a $A$ \emph{is-a} $B$: pass an $A$ where a $B$ is expected
  \item \emph{E.g.,} allow list processing functions to work on arrays
  by defining \texttt{car}, \texttt{cdr}, and \texttt{cons} for arrays
\end{itemize}
\item Declarative delegation/coercion to simulate inheritance
  \begin{itemize}
  \item Declare a function $f: a \to b$ to convert an $a$ to a $b$
    when calling specific procedures
  \item Gives restricted form of inheritance in
  \end{itemize}
\end{itemize}
\end{frame}


%-----------------------------------------------------------------------%
\begin{frame}
\frametitle{Resources}
\begin{itemize}
\item A \emph{resource} is data that is available to be used and/or
  defined in parts of the program
\item Similar to State and IO monads
\item Not explicitly passed as parameters
\item But specified in procedure declaration
\item Useful for data that is widely used/modified in a module
\item And calls to procedures that use resources must be preceded with
  \texttt{!} to signify that they use some resources
\item I/O, command line arguments are resources
\item \texttt{def hello(name) with io:} \\
\hspace*{2em}\texttt{!print("Hello, ")} \\
\hspace*{2em}\texttt{!print(name)} \\
\end{itemize}
\end{frame}


%-----------------------------------------------------------------------%
\begin{frame}
\frametitle{Delegation}
\begin{itemize}
\item 
\end{itemize}
\end{frame}


%-----------------------------------------------------------------------%
\begin{frame}
\frametitle{Type classes}
\begin{itemize}
\item 
\end{itemize}
\end{frame}



\end{document}

% $Header: /home/schachte/test/cvs-conversion/CVSROOT/research/frege/intro.tex,v 1.3 2008/11/20 11:27:49 schachte Exp $

\documentclass[12pt]{beamer}
\usepackage{alltt}
\usepackage{xspace}

% \newcommand{\lxor}{\ensuremath{\oplus}}

\newcommand{\frege}{\textsf{Frege}\xspace}

\mode<presentation>
{
  % \usetheme{Singapore}
  % \usetheme{default}
 % \usetheme{Hannover}
 % \usetheme{Goettingen}
 % \usetheme{Madrid}
% \usetheme{Copenhagen}

  % or ...

\useinnertheme{circles}
%\useoutertheme{shadow}
\setbeamercolor{normal text}{bg=yellow!25!red!3}
\usecolortheme{seahorse}
\usecolortheme{rose}

% \setbeamertemplate{blocks}[rounded][shadow=true]
% \setbeamertemplate{footline}{%
%   \hspace*{2em} \insertshorttitle \hfill
%   \insertframenumber \hfill
%   \insertsection \hspace*{2em}
% }
\setbeamercovered{transparent=5}

% \usepackage[english]{babel}
% % or whatever

% \usepackage[latin1]{inputenc}
% % or whatever

%\usepackage{times}
%\usepackage[T1]{fontenc}
% Or whatever. Note that the encoding and the font should match. If T1
% does not look nice, try deleting the line with the fontenc.
}

\title{Frege}

\subtitle{\textsf{F}unctions, \textsf{RE}lations, and 
pervasive \textsf{GE}nerics}

\author{Peter Schachte}
% % - Use the \inst{?} command only if the authors have different
% %   affiliation.

\institute[Univ.\ Melbourne] % (optional, but mostly needed)
{
  Department of Computer Science and Software Engineering \\
  University of Melbourne
}
% - Use the \inst command only if there are several affiliations.
% - Keep it simple, no one is interested in your street address.

\date{}

% \subject{}
% This is only inserted into the PDF information catalog. Can be left
% out. 



% If you have a file called "university-logo-filename.xxx", where xxx
% is a graphic format that can be processed by latex or pdflatex,
% resp., then you can add a logo as follows:

% \pgfdeclareimage[height=0.5cm]{university-logo}{university-logo-filename}
% \logo{\pgfuseimage{university-logo}}

\begin{document}

%%%%%%%%%%%%%%%%%%%%%%%%%%%%%%%%%%%%%%%%%%%%%%%%%%%%%%%%%%%%%%%%
\begin{frame}
  \titlepage
\end{frame}

%%%%%%%%%%%%%%%%%%%%%%%%%%%%%%%%%%%%%%%%%%%%%%%%%%%%%%%%%%%%%%%%
\begin{frame}{Overview}
  \tableofcontents
\end{frame}

% Since this a solution template for a generic talk, very little can
% be said about how it should be structured. However, the talk length
% of between 15min and 45min and the theme suggest that you stick to
% the following rules:  

% - Exactly two or three sections (other than the summary).
% - At *most* three subsections per section.
% - Talk about 30s to 2min per frame. So there should be between about
%   15 and 30 frames, all told.

%%%%%%%%%%%%%%%%%%%%%%%%%%%%%%%%%%%%%%%%%%%%%%%%%%%%%%%%%%%%%%%%
\section{Goals}
%%%%%%%%%%%%%%%%%%%%%%%%%%%%%%%%%%%%%%%%%%%%%%%%%%%%%%%%%%%%%%%%
\begin{frame}{\LARGE Goals of the \frege language}
  \begin{itemize}
  \item Simple and intuitive enough to teach as a first or second language
  \item Powerful and efficient enough for mainstream programming
  \item Maintainable and reusable code for large scale projects
  \item Suitable for parallel and concurrent programming
  \end{itemize}
\end{frame}


%%%%%%%%%%%%%%%%%%%%%%%%%%%%%%%%%%%%%%%%%%%%%%%%%%%%%%%%%%%%%%%%
\begin{frame}{\LARGE Simple and Intuitive}
  \begin{itemize}
  \item Simple execution model
  \item Allow imperative expression
    \begin{itemize}
    \item Loops
    \item Assignment statements
    \item Functions
    \item Procedures executed for side-effect
    \end{itemize}
  \item Objects do not change unexpectedly
  \item Fine control over object modification
  \end{itemize}
\end{frame}


%%%%%%%%%%%%%%%%%%%%%%%%%%%%%%%%%%%%%%%%%%%%%%%%%%%%%%%%%%%%%%%%
\begin{frame}{\LARGE Powerful and Efficient}
  \begin{itemize}
  \item High-level
  \item Powerful looping construct
  \item Powerful type/module system
  \item Low-overhead interface to low-level languages
  \item Optimising compiler
  \end{itemize}
\end{frame}


%%%%%%%%%%%%%%%%%%%%%%%%%%%%%%%%%%%%%%%%%%%%%%%%%%%%%%%%%%%%%%%%
\begin{frame}{\LARGE Maintainable and Reusable}
  \begin{itemize}
  \item Abstract datatypes
  \item Information hiding
  \item Resistant to programming errors
  \end{itemize}
\end{frame}


%%%%%%%%%%%%%%%%%%%%%%%%%%%%%%%%%%%%%%%%%%%%%%%%%%%%%%%%%%%%%%%%
\section{Fungible Objects}
%%%%%%%%%%%%%%%%%%%%%%%%%%%%%%%%%%%%%%%%%%%%%%%%%%%%%%%%%%%%%%%%
\begin{frame}{\LARGE Fungible Objects}
  \begin{block}{Definition \hfill \emph{\small dictionary.com}}
    \textbf{fungible:}
    \emph{(esp. of goods) being of such nature or kind as
      to be freely exchangeable or replaceable, in whole or in part, for
      another of like nature or kind.} 
  \end{block}
  \begin{itemize}[<+->]
  \item Fungible objects:  one object may be replaced with an identical one
    without changing program behaviour
  \item No object identity
  \item Changing an object cannot affect any other object
  \item Language manages aliasing for you
  \item Object equality always means deep equality
  \item No need for object cloning
  \end{itemize}
\end{frame}


%%%%%%%%%%%%%%%%%%%%%%%%%%%%%%%%%%%%%%%%%%%%%%%%%%%%%%%%%%%%%%%%
\begin{frame}{\LARGE Fungible Objects (2)}
  \begin{itemize}
  \item Variable assignment treated as a \textsf{let} that scopes the rest of
    the function definition
  \item Field assignment treated as updating variable to hold nearly
    identical object
  \item Objects passed by address, but generally cannot be modified by
    function
  \item By default, function arguments are immutable
  \item Function can declare some arguments mutable
  \item Function \emph{call} can permit or forbid argument modification
  \end{itemize}
\end{frame}


%%%%%%%%%%%%%%%%%%%%%%%%%%%%%%%%%%%%%%%%%%%%%%%%%%%%%%%%%%%%%%%%
\section{Type System}
%%%%%%%%%%%%%%%%%%%%%%%%%%%%%%%%%%%%%%%%%%%%%%%%%%%%%%%%%%%%%%%%
\begin{frame}{\LARGE Type System}
  \begin{itemize}
  \item \frege is strongly typed
  \item Each type declaration specifies an \emph{interface} (type class), as
    well as an implementation (except for abstract methods)
  \item Any type may declare that it \emph{implements} other types, if it has
    methods for all their public operations
  \item Anywhere a specific type is needed, any type that implements it may
    be passed
  \item Meyer's Uniform Access Principle: constructors and field
    access/update are just methods
  \item Types can be parametric --- generics
  \end{itemize}
\end{frame}


%%%%%%%%%%%%%%%%%%%%%%%%%%%%%%%%%%%%%%%%%%%%%%%%%%%%%%%%%%%%%%%%
\begin{frame}{\LARGE Object Oriented Features}
  \begin{itemize}
  \item No direct implementation inheritance
  \item Types may delegate messages:  convenient notation for defining
    $f(a, b, c, \ldots) = f(g(a), b, c, \ldots)$
  \item Must list messages to delegate
  \item Avoids fragile base class problem
  \end{itemize}
\end{frame}


%%%%%%%%%%%%%%%%%%%%%%%%%%%%%%%%%%%%%%%%%%%%%%%%%%%%%%%%%%%%%%%%
\end{document}

    %% \begin{itemize}
    %% \item \emph{Prevent} many coding errors
    %% \item Compile-time errors/warnings for many more
    %% \end{itemize}


% LocalWords:  insts semidet choicepoint

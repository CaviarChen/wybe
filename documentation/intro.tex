%  Emacs    : -*- latex -*-
%  File     : intro.tex
%  Author   : Peter Schachte
%  Purpose  : Introduction to Wybe programming
%  Copyright: � 2018 Peter Schachte.  All rights reserved.
%

\documentclass{article}
\usepackage{a4wide}
\usepackage{xspace}
\usepackage{fancyvrb}
\usepackage{color}
\usepackage{hyperref}
%\usepackage{ensuremath}

\newcommand{\lang}{\textsf{Wybe}\xspace}
\newcommand{\Lang}{\textsf{Wybe}\xspace}

\newcommand{\ie}{\emph{i.e.}\xspace}
\newcommand{\eg}{\emph{e.g.}\xspace}
\newcommand{\Ie}{\emph{I.e.}\xspace}
\newcommand{\Eg}{\emph{E.g.}\xspace}
\newcommand{\etc}{\emph{etc}}

\newcommand{\nyi}[1]{\underline{#1} \emph{(not yet implemented)}}
\newcommand{\tbi}[1]{(\underline{Currently}, #1).}
\newcommand{\possible}[1]{\emph{Possible improvement:} #1}

\newcommand{\tuple}[1]{\ensuremath{\langle#1\rangle}}

\DefineVerbatimEnvironment%
{wybecode}{Verbatim}
{numbers=left,frame=single,framerule=1.5pt,rulecolor=\color{blue},commandchars=@[]}

\DefineVerbatimEnvironment%
{shell}{Verbatim}
{frame=single,commandchars=@[]}


\title{Introduction to \Lang Programming}
\author{Peter Schachte}
\begin{document}
\maketitle
\noindent
\Lang is a new programming language intended to combine the best features of
declarative and imperative programming.  It is rather simple, but quite
safe and powerful.

\section{Hello, World}

To produce a \lang program, create a file whose name ends in \texttt{.wybe}.
Ordinary \lang statements at the top level of the file (i.e., not within a
function or procedure definition) are executed when the file is compiled
and run.  For example, a \lang program to print "Hello, World" would be
written like this:


\begin{wybecode}
!println("Hello, world!")
\end{wybecode}

The \texttt{println} procedure prints out a string followed by a newline
character.  The exclamation point at the beginning of the line tells \lang
that the statement performs input/output.  This will be explained in
greater depth in the section on \hyperref[Resources]{Resources}.

Multiple statements may be written, and they will be executed in order.
Comments begin with a hash (\verb`#`) character and extend to the end of the line.

\begin{wybecode}
!print("Hello, ")     # print the first word
!println("world!")    # print the rest and end the line
\end{wybecode}

\section{Compiling a \lang program}
\label{Compiling}

To compile a \lang program from a command line, use the program \texttt{wybemk} and
specify the \emph{output} file you wish to produce.  For example, if either of
the above programs is put into a file named \texttt{hello.wybe}, it can be
compiled using the command

\begin{shell}
@textbf[%] wybemk hello
\end{shell}

to produce an executable file named \texttt{hello}.  This works regardless of the
number of files and libraries used in the project.

Ordinarily, \texttt{wybemk} will simply do its job without printing anything. If
it cannot produce the requested output, it will print error messages
describing the problem. If all goes to plan, this file can then be executed
at the command line:

\begin{shell}
@textbf[%] ./hello
Hello, world!
\end{shell}

The compiler can also be used to produce an object (\texttt{.o}) file rather
than an executable file by specifying that on the command line:

\begin{shell}
@textbf[%] wybemk hello.o
\end{shell}

\section{Variables and assignment}
\label{Variables}

Variables in \lang begin with a letter (upper or lower case), and follow with
any number of letters, digits, and underscore (\verb`_`) characters, like any
number of other programming languages. What is unlike other languages is that a
variable is only assigned by a statement if it is immediately preceded by a
question mark (?). A variable may also be both used and assigned by a statement
by preceding it with an exclamation point (!), as we will see below. If not
preceded by either of these, the expression only uses the variable.

For example:

\begin{wybecode}
?msg = "Hello, world!"
!println(msg)
\end{wybecode}

The first statement assigns to the variable \texttt{msg} because it has a
question mark before it, \emph{not} because it appears to the left of the equal
sign. The second uses \texttt{msg} because it is not preceded by a question
mark. This code could equally well have been written as:

\begin{wybecode}
"Hello, world!" = ?msg
!println(msg)
\end{wybecode}

Had the first statement omitted the question mark, it would instead have tested
if \texttt{msg} is equal to \texttt{"Hello, world!"}. But the first mention of a
variable \emph{must} assign it, so this would be an error, and would be reported
as an uninitialised variable by the \lang compiler.

\section{Calling procedures and functions}
\label{Calling}
\section{Defining procedures and functions}
\label{Procedures}
\section{Control statements}
\label{Control}
\subsection{Conditional statements}
\label{Conditional}
\Lang's conditional statement has the form:

\begin{wybecode}
if { @emph[test] :: @emph[stmts]
   | @emph[test] :: @emph[stmts]
   | @emph[test] :: @emph[stmts]
   @vdots
}
\end{wybecode}
Each \emph{stmts} parts can include as many statements as needed. You may have
as many \emph{test} \texttt{::} \emph{stmts} pairs as you like; \lang will
execute the \emph{stmts} following the first \emph{test} that succeeds, and
ignore all later tests and their corresponding statements. If no \emph{test}
succeeds, the \texttt{if} statement will complete without executing any
\emph{stmts}.

\subsection{Looping statements}
\label{Loops}

\Lang has only one looping construct:

\begin{wybecode}
do {@emph[stmt]
    @emph[stmt]    
    @emph[stmt]    
    @vdots
}
\end{wybecode}
The sequence of \emph{stmt}s can contain any normal \lang statements.
In addition, the following special constructs can be used:

\begin{description}
\item[\texttt{break}]
  immediately exits the loop.

\item[\texttt{continue}]
  immediately returns to the top of the loop and begins the next iteration.

\item[\texttt{while} \emph{test}]
  immediately exits the loop unless \emph{test} succeeds.
  Equivalent to \verb`if { not` \emph{test} \verb`::` \verb`break }`

\item[\texttt{until} \emph{test}]
  immediately exits the loop if \emph{test} succeeds. Equivalent to
  \verb`if {` \emph{test} \verb`:: break }`

\item[\texttt{when} \emph{test}]
  continues executing the loop body only if the
  \emph{test} succeeds; otherwise it immediately returns to the top of the loop.
  Equivalent to \verb`if { not` \emph{test} \verb`:: continue }`

\item[\texttt{unless} \emph{test} ]
  continues executing the loop body unless the
  \emph{test} succeeds; otherwise it immediately returns to the top of the loop.
  Equivalent to \verb`if {` \emph{test} \verb`:: continue }`
\end{description}

For example, a traditional \texttt{while} loop has the form:

\begin{wybecode}
do { while @textit[test]
     @emph[stmts]
}
\end{wybecode}
and a traditional \texttt{do} ... \texttt{while} loop has the form:

\begin{wybecode}
do { @emph[stmts]
     while @emph[test]
}
\end{wybecode}

Alternatively, the \texttt{while} can be placed in the middle of the loop, in
which case, the code before the \texttt{while} will be executed one or more
times, while the code after the \texttt{while} will be executed zero or more
times. Also, \texttt{until} can be used to invert the sense of the test. For
example:

\begin{wybecode}
do { @textrm[print prompt]
     @textrm[read user input]
     until @textrm[user input is valid]
     @textrm[print admonition about what input is valid]
}
\end{wybecode}

will repeatedly print a prompt and read user input until valid input is
received, printing a message indicating that the input was invalid each time it
is invalid.

\section{Modules}
\label{Modules}
The role of modules is to present an interface to other modules, and to allow
that interface to be implemented.
Each module may define any number of functions, procedures, types, and
resources.
Each of these may optionally be declared to be part of the module's
interface by preceding its declaration with the keyword ``\texttt{public}''.
Any element not declared \texttt{public} is not visible outside the module.

Each \lang source file is automatically a module whose name is the same as the
source file, with the ``\texttt{.wybe}'' extension omitted.  Module names
(and hence source file names) may
include letters (upper and lower case), digits, and underscore characters.

A module may import other modules, making all their public elements accessible,
with the declaration
\begin{wybecode}
    use @emph[modspec]
\end{wybecode}
where \emph{modspec} names the module to import.
Multiple modules may be imported with a single declaration by separating their
names with commas.
Module names are specified relative to the current module; for example,
\texttt{use~foo} imports the sibling module \texttt{foo}.
That is, a \texttt{use} declaration begins looking for the specified module in
the same directory as the current file.
If the module is not found there, it is looked for in the library.

Modules may be nested in either of two ways.
First, a directory containing \lang source files is considered to be a module.
Second, a module may contain a declaration of the form:
\begin{wybecode}
  module @emph[name] is
  \ldots
  end
\end{wybecode}
Between the \texttt{is} and \texttt{end} keywords, all elements are considered
to be in the \emph{name} submodule of the module containing that declaration.

By default, submodules do not import anything from their supermodules, but
this can be done explicitly with the declaration:
\begin{wybecode}
    use ^
\end{wybecode}
where \texttt{\^} is a special syntax referring to the supermodule.
Similarly the supermodule's supermodule can be specified as \texttt{\^.\^},
and so on.


\section{Type declarations}
\label{Types}
\section{Tests}
\label{Tests}
\section{Resources}
\label{Resources}
\section{Interfacing to foreign code}
\label{Foreign}
\section{The \lang library}
\label{Library}
\section{Contributors}
\label{Contributors}
The following people have contributed to the design and implementation of the
\lang language:
\begin{itemize}
\item Peter Schachte
\item Ashutosh Rishi Ranjan
\item Ting Lu
\end{itemize}
\end{document}

%  Emacs    : -*- latex -*-
%  File     : compiler.tex
%  Author   : Peter Schachte
%  Origin   : Fri Mar 28 19:36:48 2003
%  Purpose  : Description of the wybe compiler architecture
%

\documentclass{article}
\usepackage{a4wide}
\usepackage{xspace}
\usepackage{fancyvrb}
%\usepackage{ensuremath}

\newcommand{\lang}{\textsf{Wybe}\xspace}
\newcommand{\Lang}{\textsf{Wybe}\xspace}

\newcommand{\ie}{\emph{i.e.}\xspace}
\newcommand{\eg}{\emph{e.g.}\xspace}
\newcommand{\Ie}{\emph{I.e.}\xspace}
\newcommand{\Eg}{\emph{E.g.}\xspace}
\newcommand{\etc}{\emph{etc}}

\newcommand{\tuple}[1]{\ensuremath{\langle#1\rangle}}

\title{The \Lang Compiler}
\author{Peter Schachte}

\begin{document}
\maketitle

\section{Aims}

The \lang compiler is designed to achieve the following goals:
\begin{itemize}
\item It should generate efficient code.
\item It should take care of compiling dependencies as needed, without
  the need for \texttt{Makefile}s or separate include files.  
\item It should compile quickly, particularly following small changes to the 
  source code.
\end{itemize}

To achieve the first goal, the compiler makes heavy use of inlining and
global analysis.
However, the use of inlining and global analysis makes the second and
third goals more difficult to achieve.

For the second goal, the compiler stores whatever is needed to compile uses
of a module its object file, and always compares modify times of source and
object files for all recursive dependencies of files being recompiled.

In addition to coding the compiler as efficiently as possible,
the third goal requires the compiler to be incremental.
We go beyond file-level incrementality, compiling individual changed
procedures, keeping the code for previously compiled procedures where possible.

Inlining and global analysis make this more difficult.
In addition to recompiling a changed dependency,
it may in general be required to recompile any
module that depends on that dependency.
This will be needed for any module that uses any changed public procedure
in the changed module, if the change has led to a change in the analysis
results, or, in the case of an inlined public procedure, in the object code.
This need for recompilation flows recursively up the module dependency
hierarchy.


\section{Architecture}
\label{sec:arch}

The compiler is split into a number of components, to manage compiler
complexity.
Many of these components are compiler passes, and others manage the
compilation process.

\subsection{Build system}
\label{sec:build}

The build system is responsible for determining which modules need to be
compiled, and oversees the compilation.
All of a module's dependencies are compiled before the module itself
is compiled.
In the case of mutually dependent modules, all the modules in each
strongly connected component in the module dependency graph are compiled
together, after all other modules it depends on.

Similarly, each SCC in the module dependency graph
(or individual non-mutually dependent module)
is compiled bottom-up, one SCC at a time, according
to the call graph.
This requires that all procedure calls are resolved prior to finding
the SCCs in preparation for compilation.


\subsection{Module resolution and type analysis}
\label{sec:call-resolution}

\Lang supports overloading of procedure calls based first on argument types,
and second based on argument modes (which arguments are inputs, which are
outputs, and which are both).
Since \lang supports optional arguments, there is no overloading based
on arity.
It is required that all procedures with the same name in single module
also have the same argument types in the same order.
They may, however, have different modes.
Thus the module of a procedure call determines the argument types, allowing
us to split call resolution into two separate phases:
first the module of each call is determined, and then the mode-overloaded
procedure within the module is determined.

Wybe semantics insists that the argument types of a procedure do not depend
on its callers, only on its definition.
Together with the problem of overloading resolution and the need to
support subtypes, this precludes the use of
the classical Algorithm W approach to type inference.
Type analysis is made more complex because
the argument types of the procedures called by a procedure
depend on which procedure those calls resolve to, while the resolution
of calls depends on the argument types
Thus module resolution must be performed concurrently with type analysis.
Resolution of modes is performed during flattening
(Section~\ref{sec:flattening}).

Type analysis is performed on a single SCC in the module dependency graph.
It is a fixed point process operating on a potential call graph.
Each node in the graph represents all procedures with the same name defined
in any of the modules being analyzed, and each edge represents a call.
Analysis precedes bottom up through this graph by strongly connected component
building a mapping from procedure definition to its argument types.
Initially the typing for each procedure in a potential call graph SCC
is approximated to be its declared type,
if it has one, or the special bottom type if not.
For each procedure definition in each node in the SCC,
we analyse the variable types in the body of the definition,
and replace the approximation of the argument types for that definition
with the lattice join of previous and newly computed parameter types.
The analysis of a single procedure body proceeds through the code collecting
a list of possible typings considering the known types of arguments for each
visible definition of each called procedure name from previously compiled
modules, plus the current approximation of the types for definitions in the
current module SCC.

This process is repeated until a fixed point is reached, and then
any type errors are reported.
If no type errors are uncovered, then a unique module is determined for
each call.
The final step is then to traverse all the procedure
bodies in the SCC, adding the determined module to each call.


\subsection{Flattening}
\label{sec:flattening}

Flattening significantly simplifies the code,
replacing all function calls with procedure calls,
eliminating all \texttt{let}, \texttt{where} and conditional
expressions, and rewriting function definitions as procedure definitions.
It also finally resolves mode-based overloading, determining for
each call which definition is intended, and splits in-out arguments
into separate inputs and outputs.


\end{document}

% LocalWords:  Makefile SCC SCCs inlining inlined wybe incrementality
% LocalWords:  subtypes subtype typing typings
